\documentclass{article}
\usepackage{color}
\usepackage{enumitem}
\usepackage{listings}

\title{Poly types}
\author{Tian Liao}
\date{\today}

\newcommand{\CodeBlockSetup}{%
  \lstset{
        language=C++,
        basicstyle=\small\ttfamily,
        keywordstyle=,
        stringstyle=,
        xleftmargin=1em,
        showstringspaces=false,
        commentstyle=\itshape\rmfamily,
        columns=fullflexible,
        keepspaces=true,
        texcl=true
  }%
}
\lstnewenvironment{codeblock}{\CodeBlockSetup}{}

\begin{document}
\pagenumbering{gobble}
\maketitle
\vfill
\begin{itemize}[noitemsep]
  \item[] Document number:
  \item[] Date: \today
  \item[] Audience:
  \item[] Authors: Tian Liao, Mingxin Wang
  \item[] Reply-to: Tian Liao \textless tilia@microsoft.com\textgreater
\end{itemize}

\newpage
\pagenumbering{arabic}

%----------------------------------------------
\section{Introduction}

%----------------------------------------------
\section{Proposal}

\subsection{Language feature}
\paragraph{}
This proposal is going to add a new keyword \textit{type} keyword to defines
type aliases that represent type erasers who keep their underlying types associated and dispatch
runtime invocations properly.\\
Here is a quick glance:
\begin{codeblock}
// define some materials.
struct Color { void apply() {} };
struct Texture { void apply() {} };
struct Glass { void apply() {} };

// declare a type alias.
type Material { void apply(); };

void foo() {
  {
    // use Material as a pointer.
    Color color;
    Material* material = color;
    material->apply();
  }
  {
    // use Material as a reference.
    Texture texture;
    Material& material = texture;
    material.apply();
  }
  {
    // host a Material in unique ptr.
    std::unique_ptr<Material> material{new Glass()};
    material->apply();
  }
}

// defining a type alias is not allowed.
void Material::apply() {} // compile error.

// instanciate a type alias is also not allowed.
Material some_material; // compile error.

\end{codeblock}

\paragraph{}
A type alias can combine other type aliases to form a new type alias.
\begin{codeblock}
type Source{ void read(); };
type Sink{ void write(); };

type DuplexStream : Source, Sink {};
type DuplexStreamEquvalent {
  void read();
  void write();
};
static_assert(std::is_same_v<DuplexStream, DuplexStreamEquvalent>);
\end{codeblock}

\paragraph{}
A type alias can declare fields.
\begin{codeblock}
type Account {
  void RefreshData();
  std::string Name;
  std::string Email;
};

class WebAccount {
  void RefreshData() { /*...*/ }
  std::string Name;
  std::string Email;
};

void consume(Account& user) {
  user.RefreshData();
  UpdateUI(user.Name, user.Email);
}

void produce() {
  WebAccount user{ .Name = "Bob", .Email = "Bob@email.com" };
  consume(user);
}
\end{codeblock}

\paragraph{}
A type alias can have function overloads.

\begin{codeblock}
type Addition {
  void operator()() const;
  int operator()(int , int) const;
  float operator()(float, float) const;
};

void foo(const Addition& add) {
  add();
  add(1, 2);
  add(0.1f, 0.2f);
}
\end{codeblock}

\paragraph{}
A type alias can be a template.
\begin{codeblock}
template <typename T, std::size_t I>
type GenericMaterial {
  using type = T;
  static constexpr std::size_t index = I;
  void apply(const T& target);
};
\end{codeblock}

\subsection{Library feature}
\paragraph{}
\begin{codeblock}
namespace std {
template <class T, size_t MaxSize, size_t MaxAlign>
class poly_ptr;
} // namespace std

void foo() {
  {
    std::poly_ptr<Material> nouse;
    assert(!nouse.has_value()); // no value.
    nouse->apply(); // undefined behavior.
  }
  {
    Glass glass;
    {
      std::poly_ptr<Material> mat = &glass; // accepts a raw pointer.
      assert(dummy.has_value()); // contains value.
      mat->apply();
    }
    glass.apply(); // glass is still alive till here.
  }
  {
    auto color = std::make_unique<Color>(); // std::unique\_ptr\textless Color\textgreater.
    std::poly_ptr<Material> mat = std::move(color); // accepts a smart ptr.
    assert(dummy.has_value()); // contains value.
    mat->apply();
  }
}
\end{codeblock}

%----------------------------------------------
\section{Motivation}

\end{document}
