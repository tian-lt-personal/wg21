\documentclass{article}
\usepackage{color}
\usepackage{enumitem}
\usepackage{listings}

\title{Enhanced typename}
\author{Tian Liao}
\date{\today}

\newcommand{\CodeBlockSetup}{%
  \lstset{
        language=C++,
        basicstyle=\small\ttfamily,
        keywordstyle=,
        stringstyle=,
        xleftmargin=1em,
        showstringspaces=false,
        commentstyle=\itshape\rmfamily,
        columns=fullflexible,
        keepspaces=true,
        texcl=true
  }%
}
\lstnewenvironment{codeblock}{\CodeBlockSetup}{}

\begin{document}
\pagenumbering{gobble}
\maketitle
\vfill
\begin{itemize}[noitemsep]
  \item[] Document number:
  \item[] Date: \today
  \item[] Audience:
  \item[] Authors: Tian Liao, Mingxin Wang
  \item[] Reply-to: Tian Liao \textless tilia@microsoft.com\textgreater
\end{itemize}

\newpage
\pagenumbering{arabic}

%----------------------------------------------
\section{Introduction}

%----------------------------------------------
\section{Proposal}

\subsection{Language feature}
\paragraph{}
This proposal is going to extend the semantics of the \textit{typename} keyword to allow it defines
type aliases that represent type erasers who keep their underlying types associated and dispatch
runtime invocations properly.\\
Here is a quick glance:
\begin{codeblock}
// define some materials.
struct Color { void apply() {} };
struct Texture { void apply() {} };
struct Glass { void apply() {} };

// declare a type alias.
typename Material { void apply(); };

void foo() {
  {
    // use Material as a pointer.
    Color color;
    Material* material = color;
    material->apply();
  }
  {
    // use Material as a reference.
    Texture texture;
    Material& material = texture;
    material.apply();
  }
  {
    // host a Material in unique ptr.
    std::unique_ptr<Material> material{new Glass()};
    material->apply();
  }
}

// defining a type alias is not allowed.
void Material::apply() {} // compile error.

// instanciate a type alias is also not allowed.
Material some_material; // compile error.

\end{codeblock}

\paragraph{}
A type alias can combine other type aliases to form a new type alias.
\begin{codeblock}
typename Source{ void read(); };
typename Sink{ void write(); };

typename DuplexStream : Source, Sink {};
typename DuplexStreamEquvalent {
  void read();
  void write();
};
static_assert(std::is_same_v<DuplexStream, DuplexStreamEquvalent>);
\end{codeblock}

\paragraph{}
A type alias can have specific constraints to control its construction, copiability, relocatability, etc.
\begin{codeblock}
typename NoCopyNoMove {
  NoCopyNoMove() = default;
  NoCopyNoMove(const NoCopyNoMove&) = delete;
  NoCopyNoMove(NoCopyNoMove&&) = delete;
};

void foo(NoCopyNoMove& a, NoCopyNoMove& b) {
  a = b; // compile error.
  a = std::move(b); // compile error.
}
\end{codeblock}

\paragraph{}
A type alias can have function overloads.

\begin{codeblock}
typename Addition {
  void operator()() const;
  int operator()(int , int) const;
  float operator()(float, float) const;
};

void foo(const Addition& add) {
  add();
  add(1, 2);
  add(0.1f, 0.2f);
}
\end{codeblock}

\paragraph{}
A type alias can be a template.
\begin{codeblock}
template <typename T, std::size_t I>
typename GenericMaterial {
  using type = T;
  static constexpr std::size_t index = I;
  void apply(const T& target);
};
\end{codeblock}

\subsection{Library feature}
\paragraph{}
\begin{codeblock}
namespace std {
template <class T>
class poly_ptr<T>;
} // namespace std

void foo() {
  {
    std::poly_ptr<Material> nouse;
    assert(!nouse.has_value()); // no value.
  }
  {
    Glass glass;
    {
      std::poly_ptr<Material> mat = &glass; // accepts a raw pointer.
      mat->apply();
      assert(dummy.has_value()); // contains value.
    }
    glass.apply(); // glass is still alive till here.
  }
  {
    auto color = std::make_unique<Color>(); // std::unique\_ptr\textless Color\textgreater.
    std::poly_ptr<Material> mat = std::move(color); // accepts a smart ptr.
  }
}
\end{codeblock}

%----------------------------------------------
\section{Motivation}

\end{document}
