\documentclass{article}
\usepackage{color}
\usepackage{enumitem}
\usepackage{listings}

\title{Enhanced typename}
\author{Tian Liao}
\date{November 25, 2023}

\newcommand{\CodeBlockSetup}{%
  \lstset{
        language=C++,
        basicstyle=\small\ttfamily,
        keywordstyle=,
        stringstyle=,
        xleftmargin=1em,
        showstringspaces=false,
        commentstyle=\itshape\rmfamily,
        columns=fullflexible,
        keepspaces=true,
        texcl=true
  }%
}
\lstnewenvironment{codeblock}{\CodeBlockSetup}{}

\begin{document}
\pagenumbering{gobble}
\maketitle
\vfill
\begin{itemize}[noitemsep]
  \item[] Document number:
  \item[] Date:
  \item[] Audience:
  \item[] Authors: Tian Liao
  \item[] Reply-to: Tian Liao \textless tilia@microsoft.com\textgreater
\end{itemize}

\newpage
\pagenumbering{arabic}

%----------------------------------------------
\section{Introduction}

%----------------------------------------------
\section{Proposal}

\paragraph{}
Quick glance
\begin{codeblock}
// declare some materials
struct Color { void apply() {} };
struct Texture { void apply() {} };
struct Glass { void apply() {} };

// define a type alias
typename Material { void apply(); };

void foo() {
  {
    // use Material as a pointer
    Color color;
    Material* material = color;
    material->apply();
  }
  {
    // use Material as a reference
    Texture texture;
    Material& material = texture;
    material.apply();
  }
  {
    // host a Material in unique ptr
    std::unique_ptr<Material> material{new Glass()};
    material->apply();
  }
}
\end{codeblock}

\paragraph{}
Type alias can also combine with other type aliases to form a new type alias.
\begin{codeblock}
typename Gettable{ void get(); };
typename Settable{ void set(); };

typename GetSet : Gettable, Settable {};
typename GetSetEquvalent {
  void get();
  void set();
};
static_assert(std::is_same_v<GetSet, GetSetEquvalent>);
\end{codeblock}

\paragraph{}
Type alias can also have specific constraints to control its copiability, relocatability, etc.
\begin{codeblock}
typename NoCopyNoMove {
  NoTrivial(const NoTrivial&) = delete;
  NoTrivial(NoTrivial&&) = delete;
};

void foo(NoCopyNoMove& a, NoCopyNoMove& b) {
  a = b; // compile error
  a = std::move(b); // compile error
}
\end{codeblock}

\paragraph{}
Function overloads.

\begin{codeblock}
typename Addition {
  void operator()() const;
  int operator()(int , int) const;
  float operator()(float, float) const;
};

void foo(const Addition& add) {
  add();
  add(1, 2);
  add(0.1f, 0.2f);
}
\end{codeblock}

%----------------------------------------------
\section{Motivation}

\end{document}
