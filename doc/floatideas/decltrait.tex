\documentclass{article}
\usepackage{color}
\usepackage{enumitem}
\usepackage{listings}

\title{decltrait}
\author{Tian Liao}
\date{\today}

\newcommand{\CodeBlockSetup}{%
  \lstset{
        language=C++,
        basicstyle=\small\ttfamily,
        keywordstyle=,
        stringstyle=,
        xleftmargin=1em,
        showstringspaces=false,
        commentstyle=\itshape\rmfamily,
        columns=fullflexible,
        keepspaces=true,
        texcl=true
  }%
}
\lstnewenvironment{codeblock}{\CodeBlockSetup}{}

\begin{document}
\pagenumbering{gobble}
\maketitle
\vfill
\begin{itemize}[noitemsep]
  \item[] Document number:
  \item[] Date: \today
  \item[] Audience:
  \item[] Authors: Tian Liao, Mingxin Wang
  \item[] Reply-to: Tian Liao \textless tilia@microsoft.com\textgreater
\end{itemize}

\newpage
\pagenumbering{arabic}

%----------------------------------------------
\section{Introduction}

%----------------------------------------------
\section{Proposal}

\subsection{Language feature}
\paragraph{} \textit{decltrait-specifier:} \\
\indent decltrait \textbf{(} \textit{class-name} \textbf{)} \\
\indent decltrait \textbf{(} \textit{class-name} \textbf{,} \textit{expression} \textbf{)} \\
where, \textit{expression} must be such that it can be evaluated as a pointer whose type conforms to the public layout defined by the \textit{class-name} type.

\textit{decltrait} returns a value that represents a fancy pointer to a polymorphic type that matches the public layout of the \textit{class-name} type.

\textit{decltrait} with the same \textit{class-name}s specified deduces a same type generated by compiler, which is a fat pointer under the hood, and by dereference which, user can see all the public member functions and public non-static data members that the \textit{class-name} type has.

\paragraph{} Example:
\begin{codeblock}
struct Drawable { void print(); };
struct Rectangle { void print() { std::println("Rectangle."); } };
struct Circle { void print() { std::println("Circle."); } };

void foo() {
  auto poly_ptr = decltrait(Drawable);
  assert(poly_ptr == nullptr); // empty because no target is assigned.

  Rectangle rectangle;
  poly_ptr = decltrait(Drawable, &rectangle);
  poly_ptr->print(); // prints "Rectangle.".

  Circle circle;
  poly_ptr = decltrait(Drawable, &circle);
  poly_ptr->print(); // prints "Circle.".

  static_assert(std::is_same_v<
    decltype(decltrait(Drawable, &rectangle)),
    decltype(decltrait(Drawable, &circle))>); // true.
}
\end{codeblock}


For the example, the following implementation is simple and dirty hack that simulates a way to achieve \textit{decltrait}:
\begin{codeblock}
#include <print>

struct Drawable {
  void print() = delete;
  void resize(int) = delete;
};
struct Rectangle {
  void print() { std::println("Rectangle."); }
};
struct Circle {
  void print() { std::println("Circle."); }
};

namespace hack {

struct DrawableDynTrait {
  virtual void print() = 0;

 protected:
  void* target;
};

struct Drawable_Rectangle : DrawableDynTrait {
  explicit Drawable_Rectangle(Rectangle* tgt) { target = tgt; }
  void print() override { static_cast<Rectangle*>(target)->print(); }
};

struct Drawable_Circle : DrawableDynTrait {
  explicit Drawable_Circle(Circle* tgt) { target = tgt; }
  void print() override { static_cast<Circle*>(target)->print(); }
};

template <class DynTrait>
class TraitPtr {
  friend struct HackFactory;

 public:
  TraitPtr() : storage_{0} {}
  TraitPtr(const TraitPtr&) = default;
  TraitPtr(TraitPtr&&) = default;
  TraitPtr& operator=(const TraitPtr&) = default;
  TraitPtr& operator=(TraitPtr&&) = default;

  DynTrait* operator->()
  { return reinterpret_cast<DynTrait*>(storage_); }
  bool has_value() const {
    return reinterpret_cast<const void*>(storage_) != nullptr;
  }
  operator bool() const { return has_value(); }
  friend bool operator==(const TraitPtr& self, std::nullptr_t)
  { return !self; }

 private:
  alignas(DynTrait) char storage_[sizeof(void*) * 2];
};

struct HackFactory {
  static auto DeclTrait_Drawable(Rectangle* target) {
    TraitPtr<DrawableDynTrait> ptr;
    new (ptr.storage_) Drawable_Rectangle{target}; // UB, but let's hack it
    return ptr;
  }

  static auto DeclTrait_Drawable(Circle* target) {
    TraitPtr<DrawableDynTrait> ptr;
    new (ptr.storage_) Drawable_Circle{target}; // UB, the same.
    return ptr;
  }
};

}  // namespace hack

int main() {
  Rectangle rect;
  Circle circle;
  // auto trait = decltrait(Drawable, \&rect);
  auto trait = hack::HackFactory::DeclTrait_Drawable(&rect);
  trait->print();  // prints Rectangle.

  // trait = decltrait(Drawable, \&circle);
  trait = hack::HackFactory::DeclTrait_Drawable(&circle);
  trait->print();  // prints Circle.
}

\end{codeblock}

%----------------------------------------------
\section{Motivation}

\end{document}
